\documentclass[fleqn]{article}
\oddsidemargin 0.0in
\textwidth 6.0in
\thispagestyle{empty}
\usepackage{import}
\usepackage{amsmath}
\usepackage[backend=bibtex]{biblatex}
\usepackage[utf8]{inputenc}
\usepackage{csquotes}
\usepackage{graphicx}
\usepackage{flexisym}
\usepackage{calligra}
\usepackage{amssymb}
\usepackage{bigints} 
\usepackage[english]{babel}
\usepackage{float}
\usepackage[colorinlistoftodos]{todonotes}
\usepackage{blindtext}
\usepackage{hyperref}

\addbibresource{references.bib}

\hypersetup{
  colorlinks=true,
  linkcolor=blue,
  filecolor=magenta,      
  urlcolor=cyan,
  pdfpagemode=FullScreen
}

\DeclareMathAlphabet{\mathcalligra}{T1}{calligra}{m}{n}
\DeclareFontShape{T1}{calligra}{m}{n}{<->s*[2.2]callig15}{}
\newcommand{\scriptr}{\mathcalligra{r}\,}
\newcommand{\boldscriptr}{\pmb{\mathcalligra{r}}\,}


\setlength{\arrayrulewidth}{0.5mm}
\setlength{\tabcolsep}{18pt}
\renewcommand{\arraystretch}{1.5}


\begin{document}

  \begin{titlepage}

    \newcommand{\HRule}{\rule{\linewidth}{0.5mm}}

    \center

    \begin{center}
      \includegraphics[height=11cm, width=11cm]{asu.png}
    \end{center}

    \vline

    \textsc{\LARGE Supernova Relic Neutrinos}\\[1.5cm]

    \HRule \\[0.5cm]
    { \huge \bfseries Formulation and Models}\\[0.4cm] 
    \HRule \\[1.0cm]

    \textbf{Behnam Amiri}

    \bigbreak

    \textbf{Prof: Cecilia Lunardini}

    \bigbreak

    \textbf{{\large \today}\\[2cm]}

    \vfill

  \end{titlepage}

  \textbf{I. Introduction}

  \vspace{10px}

  A neutrino is a particle! It’s one of the so-called fundamental particles, which means it isn’t made of any 
  smaller pieces, at least that we know of. Physicist Enrico Fermi popularized the name \textbf{neutrino}, 
  which is Italian for \emph{little neutral one}. The neutrino is so named because it is electrically neutral 
  and because its rest mass is so small that it was long thought to be zero. \textcite{One}

  They are teeny, tiny, nearly massless particles that travel at near-light speeds. Born from violent astrophysical events 
  like exploding stars and gamma-ray bursts, they are fantastically abundant in the universe, and can move as easily through 
  lead as we move through the air. But they are notoriously difficult to pin down. Neutrinos are really pretty strange particles 
  when you get down to them. they are almost nothing and because they have nearly no mass and no electric charge, sometimes, 
  we refer to them as \emph{Ghost particles}.

  If you hold your hand toward the sunlight for one second, about a billion neutrinos from the sun will pass through it. Is not that crazy? 
  At this point we might ask ourselves, why are we studying them? Well They are important to understand since they provide data of the kind 
  of processes that go on in the sun, and also a critical building block for the blueprint of nature.

  \vspace{20px}

  \textbf{II. Background Information}

  \vspace{10px}

  Initially, Particle physicists thought that neutrinos were massless. In the 1990s, a group of Japanese scientists discovered that they 
  actually have a smidgen of mass. This tiny bit of mass may explain why the universe is made up of matter, not antimatter. Early in the 
  process of the Big Bang, there were equal amounts of matter and antimatter. \textcite{Two}

  Studying neutrinos is difficult. They are tough to detect since they interact so weakly with other particles. But the newly completed 
  IceCube Neutrino Observatory will study neutrinos inside a cubic kilometer block of ice in Antarctica. Here is how it works: 
  As neutrinos pass through and interact, they produce charged particles, and the charged particles traveling through the ice give off light.
  That’s how they’re detected. It’s like having a telescope for neutrinos underground.

  Fermilab National Laboratory has an experiment that hurls a beam of neutrinos 400 miles underground from Wisconsin to Northern Minnesota 
  in about two milliseconds, and the lab is also planning a massive linear accelerator called Project X that will study the subatomic 
  particles by sending them even farther.

  \vspace{20px}

  \textbf{III. Diffuse Supernova Neutrinos}

  \vspace{10px}

  Stars with masses larger than $\approx 8M_{\bigodot}$ ($\bigodot$ is solar mass) (with $M_{\bigodot}=1.99 \times 10^{30} ~ kg$)
  end their lives with the gravitational collapse of their core, followed first by neutrino emission over a time scale of about 
  $10 s$, and then by a shock-driven, very luminous, explosion called a supernova (SN).  
  A core-collapse supernova explosion is one of the most spectacular events in astrophysics,
  and it attracts a great deal of attention from many physicists and astronomers. It
  also produces a number of neutrinos and $99\%$ of its gravitational binding energy is
  transformed to neutrinos.

  Let’s learn about two new concepts. Comoving and proper distances! In cosmology, comoving distance and proper distance are two pretty 
  close related distance measurements. \emph{Proper distance roughly corresponds to where a distant object would be at a specific moment of 
  cosmological time, which can change over time due to the expansion of the universe. Comoving distance factors out the expansion of the 
  universe, giving a distance that does not change in time due to the expansion of space.} \textcite{Three} Comoving distance and proper distance are 
  defined to be equal at the present time. At other times, the Universe's expansion results in the proper distance changing, while 
  the comoving distance remains constant.

  Prior to discuss the meaning of \emph{comoving volume} we need to get our feet wet with \emph{Hubble's law} which is observation in physical cosmology 
  that galaxies are moving away from Earth at speeds proportional to their distance. In other words, the farther they are the faster they are moving 
  away from Earth. The Hubble's law is described by the equation 
  $$v=H_0 D$$
  with $v$ velocity of a galaxy (speed of separation) in $km/s$, $H_0$ the constant of proportionality (Hubble constant) measured in $km/s/Mpc$ and between the 
  "proper distance" $D$ to a galaxy in $Mpc$ (mega-parsecs), which can change over time, unlike the comoving distance.

  The \emph{Hubble flow} describes the motion of galaxies due solely to the expansion of the Universe. The comoving volume $V_C$ is the volume in which 
  the number densities of non-evolving objects locked into the Hubble flow are constant with redshift. It is the proper volume times three factors of 
  the relative scale factor from the present to the time at redshift $z$, or $\left(1+z\right)^3$.It can be thought of as the cosmological volume with the 
  expansion of the Universe factored out, i.e. the comoving volume defined by a fixed set of objects remains constant, and is useful in cosmological calculations. 

  \vspace{20px}

  \textbf{IV. Formulation}

  \vspace{10px}

  $
    dn_v(E_v)=R_{SN}(z)\left(1+z\right)^3 \dfrac{dt}{dz} dz \dfrac{dN_v (E_v^')}{dE_v^'} dE_v^' \left(1+z\right)^{-3}
    , ~~~~~ E_v^'=(1+z) E_v
    \\
    \\
    \\
    dn_v(E_v)=R_{SN}(z)\left(1+z\right)^3 \dfrac{dt}{dz} dz \dfrac{dN_v (E_v^')}{dE_v^'} d\left[(1+z) E_v\right] \left(1+z\right)^{-3}
    \\
    \\
    \\
    \therefore ~~~ dn_v(E_v)=R_{SN}(z) \dfrac{dt}{dz} dz \dfrac{dN_v(E_v^')}{dE_v^'} \left(1+z\right) dE_v
    \\
    \\
    \begin{cases}
      E_v^'=(1+z) E_v ~~~~ \text{The energy of neutrinos at redshift} ~ z
      \\
      \\
      R_{SN}(z) ~~~~~~~~~~~~~~~ \text{Represents the supernova rate per comoving volume at} ~ z
      \\
      \\
      \dfrac{dN_v}{dE_v} ~~~~~~~~~~~~~~~~~~~ \text{The number spectrum of neutrinos emitted by one supernova explosion}
      \\
      \\
      \left(1+z\right)^{-3} ~~~~~~~~~~~~ \text{The expansion of the universe}
      \\
      \\
      z ~~~~~~~~~~~~~~~~~~~~~~~~ \text{The cosmological redshift}
      \\
      \\
      \dfrac{dF_v}{dE_v} ~~~~~~~~~~~~~~~~~~~ \text{The differential number flux of SRNs}
    \end{cases}
  $

  We know that $\dfrac{dF_v}{dE_v}=c \dfrac{dn_v}{dE_v}$ and $\dfrac{dz}{dt}=-H_0(1+z) \sqrt{\Omega_m(1+z)^3+\Omega_{\Lambda}}$ so we have the following:
  
  $
    \\
    \\
    \dfrac{dF_v}{dE_v}=c \dfrac{d}{dE_v} \left[R_{SN}(z) \left(\dfrac{-1}{H_0(1+z) \sqrt{\Omega_m(1+z)^3+\Omega_{\Lambda}}}\right) dz \dfrac{dN_v(E_v^')}{dE_v^'} \left(1+z\right) dE_v\right]
    \\
    \\
    \\
    \therefore ~~~ \dfrac{dF_v}{dE_v}
    =\dfrac{c}{H_0} \bigints\limits_{0}^{z_{max}} R_{SN}(z) \dfrac{dN_v(E_v^')}{dE_v^'} \dfrac{1}{\sqrt{\Omega_m(1+z)^3+\Omega_{\Lambda}}} dz
  $
  \\
  \\
  \\
  We assume that gravitational collapses began at the redshift $z_{max}=5$. We have the functional form for the SFR per unit comoving volume as 
  \\
  \\
  $
    \psi_{\ast}=f_{\ast} \dfrac{32}{100} \dfrac{exp(3.4 z)}{exp(3.8 z)+45} \dfrac{\sqrt{\Omega_m(1+z^3)+\Omega_{\Lambda}}}{\left(1+z\right)^{3/2}} M_{\bigodot} ~ yr^{-1} ~ Mpc^{-3}
  $
  \\
  \\
  Note: \emph{parsec (pc)} is a unit for expressing distances to stars and galaxies. It represents the distance at which the radius of Earth’s orbit 
  subtends an angle of one second of arc.
  \\
  \\
  $
    R_{SN}(z)=\dfrac{
      \bigintssss\limits_{8 M_{\bigodot}}^{125 M_{\bigodot}} dm \phi(m) 
    }{
      \bigintssss\limits_{0 M_{\bigodot}}^{125 M_{\bigodot}} dm m \phi(m) 
    } \psi_{\ast}
    \\
    \\
    =\dfrac{
      \bigintssss\limits_{8 M_{\bigodot}}^{125 M_{\bigodot}} dm \phi(m) 
    }{
      \bigintssss\limits_{0 M_{\bigodot}}^{125 M_{\bigodot}} dm m \phi(m) 
    } f_{\ast} \dfrac{32}{100} \dfrac{exp(3.4 z)}{exp(3.8 z)+45} \dfrac{\sqrt{\Omega_m(1+z^3)+\Omega_{\Lambda}}}{\left(1+z\right)^{3/2}} M_{\bigodot} ~ yr^{-1}
  $

  \clearpage

  \begin{tabular}{ |p{2cm}|p{9cm}|  }
    \hline
    \multicolumn{2}{|c|}{Glossary} \\
    \hline
      Redshift & The wavelength of the light is stretched, so the light is seen as \emph{shifted} 
      towards the red part of the spectrum.
      \\
      TDE & \emph{The tidal disruption} occurs when a star approaches sufficiently close to a supermassive 
      black hole and is pulled apart by the black hole's tidal force, experiencing spaghettification 
      \\
      SMBH & Supermassive black holes  
      \\
      GRB & Gamma-Ray Burst
      \\
      SRN & Supernova relic neutrinos
      \\
      SFR & Star formation rate
      \\
      UV
      \\
    \hline
  \end{tabular}

  \pagebreak

  \printbibliography
  

\end{document}
