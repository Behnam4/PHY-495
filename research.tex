\documentclass[fleqn]{article}
\oddsidemargin 0.0in
\textwidth 6.0in
\thispagestyle{empty}
\usepackage{import}
\usepackage{amsmath}
\usepackage{graphicx}
\usepackage{flexisym}
\usepackage{calligra}
\usepackage{amssymb}
\usepackage{bigints} 
\usepackage[english]{babel}
\usepackage[utf8x]{inputenc}
\usepackage{float}
\usepackage[colorinlistoftodos]{todonotes}


\DeclareMathAlphabet{\mathcalligra}{T1}{calligra}{m}{n}
\DeclareFontShape{T1}{calligra}{m}{n}{<->s*[2.2]callig15}{}
\newcommand{\scriptr}{\mathcalligra{r}\,}
\newcommand{\boldscriptr}{\pmb{\mathcalligra{r}}\,}

\definecolor{hwColor}{HTML}{1a0252}

\begin{document}

  \begin{titlepage}

    \newcommand{\HRule}{\rule{\linewidth}{0.5mm}}

    \center

    \begin{center}
      \includegraphics[height=11cm, width=11cm]{asu.png}
    \end{center}

    \vline

    \textsc{\LARGE Supernova Relic Neutrinos}\\[1.5cm]

    \HRule \\[0.5cm]
    { \huge \bfseries Formulation and Models}\\[0.4cm] 
    \HRule \\[1.0cm]

    \textbf{Behnam Amiri}

    \bigbreak

    \textbf{Prof: Cecilia Lunardini}

    \bigbreak

    \textbf{{\large \today}\\[2cm]}

    \vfill

  \end{titlepage}

  % \begin{enumerate}
  %   \item Start here
  % \end{enumerate}
  A core-collapse supernova explosion is one of the most spectacular events in astrophysics,
  and it attracts a great deal of attention from many physicists and astronomers. It
  also produces a number of neutrinos and $99\%$ of its gravitational binding energy is
  transformed to neutrinos.

  \vspace{20px}

  \textbf{Formulation}

  \vspace{10px}

  $
    dn_v(E_v)=R_{SN}(z)\left(1+z\right)^3 \dfrac{dt}{dz} dz \dfrac{dN_v (E_v^')}{dE_v^'} dE_v^' \left(1+z\right)^{-3}
  $
  \\
  \\
  Since $E_v^'=(1+z) E_v$ we have
  \\
  \\
  $
    =R_{SN}(z)\left(1+z\right)^3 \dfrac{dt}{dz} dz \dfrac{dN_v (E_v^')}{dE_v^'} d\left[(1+z) E_v\right] \left(1+z\right)^{-3}
    \\
    \\
    \\
    \therefore ~~~ dn_v(E_v)=R_{SN}(z) \dfrac{dt}{dz} dz \dfrac{dN_v(E_v^')}{dE_v^'} \left(1+z\right) dE_v
  $
  \\
  \begin{description}
    \item[$E_v^'=(1+z) E_v:$ The energy of neutrinos at redshift $z$]
    \item[$R_{SN}(z):$ Represents the supernova rate per comoving volume at $z$]
    \item[$\dfrac{dN_v}{dE_v}:$ The number spectrum of neutrinos emitted by one supernova explosion]
    \item[$\left(1+z\right)^{-3}$:  The expansion of the universe] 
  \end{description}

  \vspace{20px}

  We know that $\dfrac{dF_v}{dE_v}=c \dfrac{dn_v}{dE_v}$ and $\dfrac{dz}{dt}=-H_0(1+z) \sqrt{\Omega_m(1+z)^3+\Omega_{\Lambda}}$ so we have the following:
  
  $
    \\
    \\
    \dfrac{dF_v}{dE_v}=c \dfrac{d}{dE_v} \left[R_{SN}(z) \left(\dfrac{-1}{H_0(1+z) \sqrt{\Omega_m(1+z)^3+\Omega_{\Lambda}}}\right) dz \dfrac{dN_v(E_v^')}{dE_v^'} \left(1+z\right) dE_v\right]
    \\
    \\
    \\
    =-\dfrac{c}{H_0} \bigints\limits_{0}^{z_{max}} R_{SN}(z) \dfrac{dN_v(E_v^')}{dE_v^'}   \dfrac{1}{\sqrt{\Omega_m(1+z)^3+\Omega_{\Lambda}}}   dz
  $

\end{document}